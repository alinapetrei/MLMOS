\documentclass[a4paper,10pt]{article}
\usepackage[utf8]{inputenc}

\title{
Bitcoin împreună cu machine learning\\
{\normalsize
Relaţia dintre preţurile Bitcoinului şi variabilele economice fundamentale
}
{\normalsize
Facultatea de informatică, UAIC
}
}
\author{Apetrei Alin-Cosmin}
\date{}

\begin{document}
\maketitle

\textbf{Introducere}\\

Bitcoin este o monedă digitală care a apărut recent ca sistem de plăți peer-to-peer pentru
facilitarea tranzacțiilor. Nu este emisă de nicio bancă centrală sau altă instituție financiară, ci utilizează
metodele criptografice și se bazează pe un algoritm de software open source care verifică descentralizarea
tranzacțiilor și controlează crearea de Bitcoins noi. Fluctuațiile mari ale prețurilor Bitcoin
(în special în anul 2013) și creșterea imensă a capitalizării pieței asociate
au dat naștere unei ramuri a literaturii studiind factorii care ajută la explicarea sau prezicerea pretului monedei Bitcoin.
\newline

\textbf{Abstract}\\

În această lucrare studiem dinamica care guvernează formarea prețurilor Bitcoin concentrându-se pe Twitter feeds
 ca factor explicativ împreună cu alte variabile economice și tehnologice.Analiza a fost efectuată zilnic prin utilizarea 
unui algoritm de învăţare automată, și anume mașinile de suport vectorial (SVM). 
Analiza pe termen scurt arată că numărul interogărilor de căutare Wikipedia (indicând gradul de
interesul public pentru Bitcoins) și rata de hash (măsurarea dificultății miniere) are un efect pozitiv asupra
prețului unui Bitcoin. Valoarea Bitcoins este afectată negativ de cursul de schimb
între USD și euro (care reprezintă nivelul general al prețurilor). Acest tip de analiză pe termen lung arată că prețul Bitcoin este asociat pozitiv cu numărul de Bitcoins în circulație (reprezentând stocul total de bani) și negativ
asociat cu indicele pieței bursiere.

\end{document}